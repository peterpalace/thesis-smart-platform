\chapter{Cos'è una Smart City\label{cap1}}
\fancyhead[RO]{\bfseries Cos'è una Smart City}

Secondo le Nazioni Unite (UN) \cite{UNProjection}, entro il 2050, il 68\% della popolazione mondiale, circa 2.5 miliardi di persone, vivrà in città. Pertanto è fondamentale chiedersi in che modo è possibile rendere questa crescita sostenibile per i cittadini, quindi migliorando i trasporti pubblici, la sicurezza e più in generale la qualità della vita, ma anche e sopratutto per l'ambiente, cercando di ridurre le emissioni e migliorando l'efficienza energetica.

Techopedia definisce \cite{SmartCityDefinition} una città intelligente come quella che incorpora le tecnologie per migliorare le funzioni della città e la qualità della vita dei suoi cittadini. La vivibilità è un fattore chiave nelle città intelligenti e quindi parte dei loro principi guida è di ridurre al minimo l'uso delle risorse, evitare sprechi e ridurre i costi complessivi. Più in generale, l'obiettivo delle smart cities è quello di migliorare la qualità della vita dei cittadini attraverso tecnologie moderne.

Negli anni sono nati vari progetti in tutto il mondo con focus vari e vision diverse: focus sul miglioramento della vita dei cittadini, focus sulla sicurezza, sull' efficientamento energetico o dei trasporti; Per Munish Khetrapal, il direttore del programma smart city e Iot in Cisco, già 5000 città sono sul cammino per diventare smart city.
La discussione su quale sia "il cammino" da intraprendere è ancora aperta, secondo lo smart cities council si possono evidenziare 4 step essenziali:
\begin{itemize}
\item Definizione degli obiettivi che si vogliono raggiungere e della vision. 
\item Definire come usare la tecnologia per migliorare vivibilità, lavoro e sostenibilità connettendo parti del governo pubblico, i cittadini e le imprese del territorio.
\item Definizione di KPI
\item Analisi dei dati ricevuti dai sistemi informatici per capire la situazione corrente e predire quale sarà la successiva.
\end{itemize}

\section{Vivere in una smart city}
Mi sveglio alle 7:00, la mia sveglia connessa inonda la camera di una luce leggera per simulare l'alba e una musica dolce mi sveglia gentilmente. Le serrande si aprono automaticamente mostrando una giornata uggiosa ma la temperatura della camera è perfetta grazie al termostato che ha autoregolato la temperatura della casa. Mentre guardo fuori, ricevo una notifica dalla caffettiera, il caffè è pronto e l'odore mi convince ad alzarmi. 
Finita colazione, guardo l'app dei bus autonomi, ne passerà uno tra 10 minuti. Alla fermata, la pensilina mi riconosce e mi propone dei biglietti per il teatro stasera, prenoto due posti. Il bus elettrico arriva, appeno entro sento il cellulare vibrare, mi è stato sottratto il costo del biglietto dal pass cittadino salvato sullo smartphone. Il pass cittadino è una carta che uso per accedere e usufruire di tanti altri servizi pubblici oltre ai trasporti. Il bus durante il tragitto recupera i dati dei passeggeri in modo che un algoritmo possa massimizzare le corse e la strada da fare il giorno successivo.
Domani prenderò la macchina elettrica del car-sharing cittadino - si trova una macchina libera vicino a te e si prenota sull'app. Il traffico non è più un problema, un'app usa i sensori installati in tutta la città e le telecamere per monitorare il traffico e indicarti la strada migliore. Usando questi dati e algoritmi predittivi la città può anticipare gli ingorghi stradali.

Questo è un esempio di come è possibile migliorare la qualità della vita vivendo in una smart city \cite{theEconomicAndSocialValue}.
Il concetto di Smart City è visto sempre di più come un servizio al cittadino e la Smartness si compone di diversi aspetti. Almeno sei secondo il portale SmartCitiesWorld: connettività, dati, energia, edilizia, trasporti e governi.

\subsection{Barcellona, Spagna}
Barcellona è una delle prime grandi aree metropolitane a diventare una città smart. Secondo Cisco\cite{theEconomicAndSocialValue}, il modello Smart City di Barcellona si concentra su 12 diverse aree, tra le quali la tecnologia ambientale, la tecnologia dell’informazione e comunicazione, la mobilità, l’acqua, l’energia, i rifiuti, lo spazio pubblico e i flussi di informazioni e servizi. La città ha 22 programmi principali che coprono queste aree, con iniziative che includono l'illuminazione intelligente e il parcheggio, così come la gestione dell'acqua e dei rifiuti. Grazie al collegamento tra i dipartimenti della città è possibile fornire servizi coordinati, mentre attraverso il sistema operativo cittadino è possibile raccogliere e analizzare rapidamente i dati raccolti dalla rete.
Barcellona ha anche realizzato Vincles BCN, un'app innovativa rivolta ai cittadini anziani che permetterà loro di rimanere in contatto con figli e altri membri della famiglia, ricevere avvisi sulle medicine e su come raggiungere medici o ospedali. Il design dell'app Vincles ha vinto il Gran Premio di Mayors Challenge di Bloomberg Philanthropies nel settembre 2014.
Ciò che rende Barcellona un modello nel panorama delle città smart è l’impatto globale della sua iniziativa. L’aspetto più sorprendente del suo approccio è l’attenzione a soluzioni che, sebbene si servono degli strumenti tecnologici, hanno come fine ultimo il miglioramento della qualità di vita degli individui  e, dunque, la centralità del cittadino, considerato il vero beneficiario delle soluzioni smart \cite{theEconomicAndSocialValue}. 

\subsection{Torino, Italia}
Torino è stata probabilmente la prima città in Italia a cominciare quel processo di trasformazione digitale che la vuole portare ad essere la prima smart city.
Nella visione torinese, una città smart mette al centro i suoi abitanti, crea i servizi che essi desiderano, rende più facile vivere e lavorare, risulta accogliente e aperta verso l’esterno \cite{SmartCityTorino}.
Per rendere tutto questo possibile, si è puntato prima di tutto sull'estensione della rete Wi-Fi cittadina per permettere anche alle zone più periferiche di usufruire dei servizi smart. 

Oltre alla comunicazione tra cittadini, ciò che ricopre un ruolo fondamentale per la riuscita del progetto, è la comunicazione tra gli oggetti. Per questo motivo sono state avviate numerose collaborazioni tra il comune e le imprese del territorio per potenziare l'infrastruttura IoT e permettere il proliferare di "servizi smart".

Un esempio da considerare è quello di Piazza Risorgimento che si propone di diventare la prima "piazza smart d'Italia" attraverso l'illuminazione intelligente, il monitoraggio dei consumi idrici ed elettrici e sensori in grado di comunicare la disponibilità nei parcheggi. Oltre all'installazione dei sensori, è stato creato un orto urbano che ha generato nuove connessioni tra gli abitanti della zona, migliorando la socialità e l'inclusione.

\subsection{Stoccolma, Svezia}
Per diventare una smart city il comune di Stoccolma sta coordinando numerosi programmi di digitalizzazione. Tutti gli investimenti sono basati sui bisogni dei cittadini che vivono la città o che la visitano da turisti.

La strategia per raggiungere gli obiettivi che la città si è prefissata è stata sviluppata insieme ai residenti, l'università e le aziende del territorio \cite{StockholmStrategy}:
\begin{itemize}
    \item La città ha organizzato riunioni nel comune, aperte a tutti, per discutere di obiettivi e bisogni
    \item I social media sono stati usati per ricevere feedback sulla vision e consigli su come implementare i numerosi progetti nella città
    \item Startup e aziende affermate sono state invitate a discutere con i dipendenti pubblici per trovare una soluzione comune.
\end{itemize}

La strategia di Stoccolma per implementare la sua vision di smart city può essere divisa in due aree fondamentali \cite{StockholmStrategy}:
\begin{itemize}
    \item \emph{Implementation}. Comprende i principi da seguire per l'implementazione della vision: Coordinazione e collaborazione, comunicazione e definizione della priorità dei progetti.
    \item \emph{Enablers}. Connettività, piattaforme integrate, sensori e altre tecnologie sono visti come elementi abilitanti per l'implementazione di una smart city.
\end{itemize}

Sono già stati sviluppati numerosi progetti nella città \cite{StockholmProjects}:
\begin{itemize}
    \item \emph{Bidoni Smart}. I cestini dell'immondizia sparsi per la città avvisano in real-time quando sono quasi pieni e bisogna svuotarli. Inoltre contengono una tecnologia che comprime i rifiuti. Solitamente i bidoni dell'immondizia devono essere svuotati 1-3 volte al giorno mentre con questa tecnologia si riesce a svuotarli 4 volte alla settimana.
    \item \emph{Suggerisci un'app}. Questo servizio consente ai cittadini di denunciare i disservizi. Quando il sistema riceve una richiesta, questa è diretta alla sezione del comune competente la quale può richiedere immediatamente ad un'azienda di risolvere il problema. In questo modo cittadini, comune e aziende sono facilmente collegate.
    \item \emph{Smart lighting}. Con questa tecnologia la luce sulle strade si accende solo se una persona o una macchina si sta avvicinando.
\end{itemize}
\subsection{Londra, Inghilterra}
Londra ha reso i cittadini consapevoli dell'importanza di avere una città smart, grazie ad uno strumento di comunicazione e dialogo come il London Datastore \cite{evolvere}.
Attraverso questo strumento i cittadini possono proporre nuove soluzioni e le aziende usare gli open data per fornire servizi a supporto della collettività. 

Gli obiettivi della città sono molteplici: produrre energia rinnovabile, emissioni zero entro il 2050, costruzione di nuovi parchi solari e dare un forte segnale di appoggio alla politica green nazionale e internazionale.

È stato inoltre istituito lo Smart London Board, un organismo appositamente creato per la mobilità di Londra.


\section{Conclusioni}
Qualunque sia il focus o le diverse fasi da intraprendere, le città che vogliono diventare intelligenti stanno cercando di coinvolgere i cittadini per renderli partecipi e protagonisti attivi delle opportunità e dei benefici offerti dalle smart cities.

Un caso interessante in questo senso è quello di Google. Nel 2012, cercava di lanciare il suo nuovo servizio di fibra a Kansas City. Inviando survey sul servizio nelle aree metropolitane il team ha scoperto che per un quarto degli intervistati internet non era rilevante per le loro vite; quindi Google decise di investire in "programmi di alfabetizzazione digitale"\cite{theEconomicAndSocialValue}. 

Questo esempio evidenzia l'importanza di partire dal cittadino per comprenderne i bisogni e sviluppare delle soluzioni che siano in grado di rispondervi.